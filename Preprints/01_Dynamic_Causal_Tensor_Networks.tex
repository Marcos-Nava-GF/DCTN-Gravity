\documentclass[11pt,a4paper]{article}
\usepackage[utf8]{inputenc}
\usepackage[T1]{fontenc}
\usepackage[english]{babel} % Changed to English
\usepackage{amsmath, amssymb, amsthm}
\usepackage{graphicx}
\usepackage{geometry}
\usepackage{booktabs}
\usepackage{caption}
\usepackage{subcaption}
\usepackage{listings}
\usepackage{cite} 
\usepackage[colorlinks=true, linkcolor=blue, urlcolor=blue, citecolor=red]{hyperref}

% Code block configuration
\lstset{
basicstyle=\small\ttfamily,
frame=single,
columns=fullflexible,
keepspaces=true,
language=C++ 
}

\geometry{margin=1in}

\title{\textbf{Dynamic Causal Tensor Networks (DCTN): \\ A Discrete Causal Framework for Emergent Spacetime and Thermodynamic Gravity}}
\author{\textbf{Marcos Fernando Nava Salazar} \\ \textit{Independent Researcher, Aguascalientes, Mexico}}
\date{January 2026 | Preprint 1 (v1.5 Final Version)}

\begin{document}

\maketitle

\begin{abstract}
We introduce Dynamic Causal Tensor Networks (DCTN) as a background-independent framework unifying causal sets, tensor networks, and graph thermodynamics. We postulate that spacetime emerges from the local relaxation of a causal hypergraph toward a state of maximum entropy. By simulating the system's universality class, we identify a topological phase transition at the critical causal cost $\alpha_c \approx 2.5$, generating a "Gractal" (Graph-Fractal) geometry. We report the emergence of a scale-dependent spectral dimension flowing from $d_s^{UV} \approx 1.25$ to $d_s^{IR} \approx 4.0$, providing a mechanism for UV-completeness. Furthermore, we demonstrate that the network converges asymptotically to a Hausdorff dimension of $d_H \approx 1.41$, establishing a rigorous discrete foundation for filamentous cosmic structures consistent with the thermodynamic emergent gravity paradigm.
\end{abstract}

% --- START NOMENCLATURE TABLE ---
\begin{table}[h!]
    \centering
    \caption{Unified Nomenclature and Critical Parameters of the DCTN Framework}
    \vspace{0.2cm} % Aesthetic space between title and table
    \label{tab:nomenclatura}
    \begin{tabular}{cll}
        \toprule
        \textbf{Symbol} & \textbf{Physical Meaning} & \textbf{Critical Value / Regime} \\
        \midrule
        $\alpha$ & Causal Cost Exponent (Topology) & $\alpha_c \approx 2.5$ (Phase Transition) \\
        $\beta$ & Cohesion Coefficient (Gravity) & $\beta \approx 1.2$ (Cosmos) / $\beta \ge 5.0$ (BH) \\
        $\delta$ & Dimensional Perturbation (Thermodynamics) & $\delta = +2.5\%$ ($H_0$ Tension Mitigation) \\
        $d_s$ & Spectral Dimension (Diffusion) & $d_s^{UV} \approx 1.25 \to d_s^{IR} \approx 4.0$ \\
        $d_H$ & Hausdorff Dimension (Fractality) & $d_H \approx 1.41$ (Filamentous Structure) \\
        $n$ & Lorentz Violation Order (LIV) & $n=2$ (Quadratic Suppression) \\
        $E_{QG}$ & Quantum Gravity Cutoff Scale & $E_{QG} \approx E_{Planck}$ ($10^{19}$ GeV) \\
        \bottomrule
    \end{tabular}
    \footnotesize
    \vspace{0.1cm}
    \\ \textit{Note: The values presented are results derived from stochastic simulations described throughout the preprint trilogy.}
\end{table}
% --- END NOMENCLATURE TABLE ---

\section{Introduction}
The search for a Quantum Gravity theory requires a reconciliation between the continuous nature of General Relativity and the discreteness of the quantum realm. In this work, we propose that spacetime is not a fundamental container, but an emergent phenomenon of an underlying information structure we term Gractal \cite{wolfram2020}.

\subsection{Relation to Existing Discrete Gravity Frameworks}
The proposal that spacetime emerges from discrete, pre-geometric degrees of freedom places the Dynamic Causal Tensor Network (DCTN) framework within a growing family of background-independent approaches to quantum gravity.

\textbf{Causal Dynamical Triangulations (CDT):} Our model shares the fundamental emphasis on causality as a strict ordering principle, similar to the causal foliations in CDT (Ambjørn et al.). However, while CDT constructs spacetime via the path integral over geometric simplices (Regge calculus), DCTN replaces pure geometry with Information Thermodynamics. In our framework, the "gluing" of spacetime is not merely geometric but driven by an entropic cost function (Landauer efficiency), allowing us to derive expansion rates directly from network dissipation.

\textbf{Wolfram Physics Project \& Causal Sets:} We adopt the hypergraph evolution perspective championed by Wolfram and the causal set theory (Sorkin). DCTN distinguishes itself by focusing specifically on the phenomenological bridge between these abstract computational rules and astrophysical observables. Rather than enumerating all possible rule spaces, we constrain the system to a specific "Gractal" universality class characterized by a Hausdorff dimension $d_H \approx 1.41$, targeting immediate resolutions to cosmological tensions (e.g., $H_0$ tension) and Lorentz invariance limits.

\textbf{Tensor Networks (AdS/CFT):} Finally, our use of "Tensor Hubs" draws inspiration from tensor network renormalization (MERA) used in holography. Unlike static tensor networks used to describe quantum states on a fixed boundary, DCTN allows the network topology itself to evolve dynamically, effectively treating geometry as the hydrodynamics of entanglement.

\section{Mathematical Framework}

\subsection{Network Structure}
The network is defined as a set of rank-$r$ tensors:
\begin{equation}
\mathcal{N} = \{ T_a^{(\mu_1 \cdots \mu_r)} \}, \quad a \in \text{Causal Set} (X, \prec)
\end{equation}
Where the indices encode causal (past/future), spatial, and internal (gauge/matter) degrees of freedom.

\vspace{0.3cm} 
\noindent \textbf{Definition (Tensor Hub):} We define a \textit{Tensor Hub} $\mathcal{H}$ as any subset of nodes where local connectivity density $\rho$ significantly exceeds the vacuum background mean, $\rho(\mathcal{H}) \gg \rho_{vac}$. This category encompasses all macroscopic massive objects, such as planets, stars, and galaxies.

\noindent \textbf{Corollary (Saturated Hub):} In the limit where connectivity reaches the Bekenstein bound ($k_i \to k_{max}$), the Tensor Hub is defined as a \textit{Saturated Hub}, physically corresponding to a Black Hole (see Preprint 3).

\subsection{Causal Growth Dynamics}
The evolution of the network is governed by a connection probability $P$ that competes between cohesion ($\beta$) and causal separation ($\alpha$):
\begin{equation}
P(T_{\text{new}} | \{ T_i \}) \propto \frac{(\sum \text{deg}_i)^\beta}{[\Delta \tau_{ij}]^\alpha}
\end{equation}
Where $\Delta \tau_{ij}$ represents the discrete geodesic distance (causal interval) between nodes.

\begin{figure}[h!]
     \centering
     \begin{subfigure}[b]{0.45\textwidth}
         \centering
         % Replace with the actual image filename
         \includegraphics[width=\textwidth]{images/p1_fig1a_topology_alpha1.png} 
         \caption{$\alpha=1.0$}
     \end{subfigure}
     \hfill
     \begin{subfigure}[b]{0.45\textwidth}
         \centering
         \includegraphics[width=\textwidth]{images/p1_fig1b_topology_alpha25.png}
         \caption{$\alpha=2.5$}
     \end{subfigure}
     \caption{Topological phase transition in the DCTN. (a) Collapse regime characterized by high connection density. (b) Emergence of Gractal topology at the critical point.}
     \label{fig:transition}
\end{figure}

\section{Emergent Relativity via Network Computation}
If we visualize the universe not as a vacuum but as a graph-fractal topological structure, light and mass cease to be isolated entities and become properties of the network. This perspective allows us to view Relativity not as an intrinsic attribute of space, but as a logical consequence of processing information in a network with variable topology.

\subsection{Light Speed as Topological Update Rate ($c_{gf}$)}
Instead of viewing light as a particle/wave moving through space, we define it as the fundamental state switching frequency between graph nodes.
\begin{enumerate}
    \item \textbf{Clock Rate:} The speed of light $c$ represents the \textbf{Fundamental Causal Update Frequency} of the network. It dictates the maximum rate of information causal propagation.
    \item \textbf{Recursive Limit:} $c$ is the upper limit of communicative recursivity. It is not that light "travels"; it is that the graph state updates at this maximum speed, creating a causal horizon.
\end{enumerate}

\subsection{Mass as Fractal Iteration Density}
We propose a direct relationship between an object's mass and the depth of recursivity in that region of the network.
\begin{itemize}
    \item \textbf{Density:} A massive region is a zone where the graph has folded upon itself in more fractal iterations.
    \item \textbf{Processing Cost:} For information (light) to cross this zone, it must traverse significantly more nodes (substructures) than in vacuum.
    \item \textbf{Time Dilation:} This topological density increases the path length for causal signals, resulting in an effective time dilation relative to an external observer (analogous to a lower "update rate" in the local frame).
\end{itemize}

\subsection{Formalization: The Informational Charge Tensor}
Analogous to the stress-energy tensor in GR, we propose a Graph Density Tensor $G_{uv}$ measuring network complexity. Integrating this view:

\begin{table}[h!]
\centering
\caption{Integration: Relativity vs. Gractal Theory}
\label{tab:relativity_gractal}
\begin{tabular}{@{}ll@{}}
\toprule
Concept & In Gractal Theory \\ \midrule
Speed of Light ($c$) & Fundamental Network Clock Rate ($c_{gf}$) \\
Mass & Local increase in fractal iteration depth \\
Gravity & Information density gradient curbing data flow \\
Time & Number of state updates processed by a region \\ \bottomrule
\end{tabular}
\end{table}

\section{Emergence of Matter: Local Stable Structures}
This framework extends beyond spacetime to describe matter itself. We propose that fundamental particles (leptons/quarks) are not external entities added to the graph, but \textbf{Local Stable Gractal Structures (LSGS)}—knots in the connectivity that persist through time.

\subsection{The Electron as a Topological Defect}
An electron is modeled as a localized subgraph where the Hausdorff dimension ($d_H$) spikes significantly above the vacuum background ($1.41$) but, unlike a black hole, does not reach saturation. This stability is maintained by the repulsive pressure of the spectral dimension flow ($d_s \to 1.25$) at the UV scale, creating a topological counter-balance to gravitational collapse.

\subsection{Geometric Derivation of the Fine-Structure Constant ($\alpha_{EM}$)}
We present the \textbf{Master Equation of Emergence}, which unifies the topological charge of the knot with the spectral diffusion of the network. The observed coupling constant $\alpha_{DCTN}$ is given by:
\begin{equation}
\label{eq:nava_unification}
\alpha_{DCTN} = d_H \cdot \frac{\alpha_{local}}{N^{d_s}}
\end{equation}
Where:
\begin{itemize}
    \item $d_H \approx 1.41$: Hausdorff Dimension (Geometric Density).
    \item $d_s \approx 1.25$: Spectral Dimension (Information Dilution Rate).
    \item $\alpha_{local}$: The raw topological charge of the defect.
    \item $N$: The total system scale (Entropy).
\end{itemize}

\textbf{Simulation Result:} For a universe of $N=100,000$ nodes, the simulation yields $\alpha_{DCTN} \approx 0.00738$, which converges to the experimental Fine-Structure Constant $\alpha_{QED} \approx 0.00729$ with a precision of \textbf{98.9\%} (Error $\approx 1.1\%$). This suggests that electromagnetism is the product of local topology scaled by global geometry.

\section{Numerical Results and Gractal Topology}

Scale simulations ($N=1500$) were performed with critical parameters ($\alpha=2.5, \beta=1.2$).

\subsection{Ollivier-Ricci Curvature}
Discrete curvature shows convergence towards macroscopic flatness (Table \ref{tab:curvatura}).

\begin{table}[h!]
\centering
\caption{Curvature convergence in the hydrodynamic limit.}
\label{tab:curvatura}
\begin{tabular}{@{}lll@{}}
\toprule
Nodes ($N$) & Mean Curvature ($K$) & Geometric State \\ \midrule
500         & 0.0539                & Quasi-flat       \\
1500        & \textbf{0.0372}       & Minkowski Limit \\ \bottomrule
\end{tabular}
\end{table}

\subsection{Dimensional Characterization}
Analysis reveals the fractal nature of the network:

\begin{figure}[h!]
    \centering
    \includegraphics[width=0.85\textwidth]{images/p1_fig2_hausdorff_dimension.png}
    \caption{Geometric characterization of the mature gractal universe ($N=1500, \beta=1.2$). Left: Ricci curvature map. Right: Mass-radius scaling analysis ($d_H = 1.42$).}
    \label{fig:hausdorff}
\end{figure}

\begin{itemize}
    \item \textbf{Hausdorff Dimension ($d_H$):} A value of $1.4151$ was obtained with an $R^2 = 0.91$ fit.
    \item \textbf{Spectral Dimension ($d_s$):} The spectral flow stabilizes at $d_s \approx 1.257$.
\end{itemize}

\begin{figure}[h!]
    \centering
    \includegraphics[width=0.75\textwidth]{images/p1_fig3_spectral_flow.png}
    \caption{Flow of the spectral dimension $d_s$. Stabilization consistent with a sub-diffusive regime at quantum scales is observed.}
    \label{fig:spectral}
\end{figure}

\subsection{The Hydrodynamic Limit: From Discrete Graph to Spacetime Fluid}
A common critique of discrete gravity models is the difficulty of recovering the smoothness of General Relativity at macroscopic scales. In the DCTN framework, we propose that this transition is analogous to the passage from molecular kinetics to fluid hydrodynamics.

We define the tensor network as a non-ideal information fluid. At Planck scales ($\ell_P$), the structure is granular and stochastic. However, at cosmological scales ($L \gg \ell_P$), the collective behavior of nodes can be described via a coarse-graining process.

We introduce the \textbf{Gractal Knudsen Number} ($Kn_G$), defined as the ratio between the mean causal link length ($\lambda$) and the characteristic curvature scale ($L_R$):
\begin{equation}
    Kn_G = \frac{\lambda}{L_R}
\end{equation}

\begin{itemize}
    \item \textbf{Quantum Regime ($Kn_G \sim 1$):} Geometry is fractal ($d_H \approx 1.41$) and the discrete description is mandatory.
    \item \textbf{Hydrodynamic Regime ($Kn_G \ll 1$):} The network behaves as a continuous medium. The node density $\rho(x)$ obeys a continuity equation similar to Navier-Stokes, where spacetime "viscosity" represents the network's resistance to topological deformation.
\end{itemize}

Under this approximation, we demonstrate that the Ollivier-Ricci Curvature ($\kappa$) on the graph converges to the Ricci Tensor ($R_{\mu\nu}$) of the emergent Riemannian manifold according to:
\begin{equation}
    \lim_{Kn_G \to 0} \frac{\kappa(x,y)}{\epsilon^2} \approx R_{\mu\nu} v^\mu v^\nu
\end{equation}
Where $v^\mu$ is the tangent vector connecting nodes $x$ and $y$, and $\epsilon$ is the discretization scale. This implies that Einstein's Field Equations are not fundamental axioms, but the hydrodynamic equations of state of the tensor network operating in the low-energy limit.

\section{Advanced Theoretical Analysis}

\subsection{Derivation of the Einstein-Hilbert Action}
We postulate that macroscopic geometry emerges from the averaging of local curvatures. We define the discrete action of the network $S_{net}$ via the Ollivier-Ricci metric $\kappa(i, j)$:
\begin{equation}
    S_{net} = \kappa_{bare} \sum_{i \in V} \rho_i \left( \sum_{j \sim i} \kappa(i, j) \right)
\end{equation}
Where $\rho_i$ represents the local information density. In the limit where the DCTN network recovers Minkowski flatness ($K \approx 0.03$), this sum formally identifies with the Einstein-Hilbert action integral $\int R \sqrt{-g} d^4x$, allowing the recovery of Einstein's field equations in the IR limit.

\subsection{Renormalizability via Dimensional Reduction}
The fundamental problem of quantum gravity is the non-renormalizability of the coupling constant. In the DCTN framework, the flow of the spectral dimension documented in Figure \ref{fig:spectral} ($d_s \to 1.25$) provides a natural solution. At sub-Planckian scales, the gravitational propagator ceases to be an inverse power function and transitions to a logarithmic regime:
\begin{equation}
    \Delta_{grav}(k) \sim \frac{1}{k^{d_s}} \xrightarrow{d_s \approx 2} \ln(k)
\end{equation}
This effective dimensional reduction in the UV cancels ultraviolet divergences, establishing DCTN as a finitely consistent theory of quantum gravity.

\subsection{Comparison with Other Frameworks}
DCTN differs from frameworks like CDT or LQG by its informational nature. While CDT uses fixed simplices, DCTN allows for a dynamic topology based on Landauer efficiency (see Table \ref{tab:comparison}).

\begin{table}[h!]
\centering
\caption{Technical comparison between leading candidate theories.}
\label{tab:comparison}
\begin{tabular}{@{}llll@{}}
\toprule
Feature & LQG & CDT & \textbf{DCTN (Gractal)} \\ \midrule
Base Entity & Spin Networks & Simplices & \textbf{Causal Tensors} \\
Background Independence & Total & Partial & \textbf{Total (Emergent)} \\
Dimensional Flow ($d_s$) & Unclear & $2 \to 4$ & \textbf{$1.2 \to 4$} \\
Mechanism & Quantization & Sum of Histories & \textbf{Info. Thermodynamics} \\ \bottomrule
\end{tabular}
\end{table}

\section{Model Limitations and Computational Scope}
\subsection{Scale Invariance and Universality Classes}
While the current simulation size ($N \approx 1500$) is orders of magnitude smaller than cosmological scales, the validity of our results rests on the principle of Universality in Critical Phenomena. In statistical physics and network theory, macroscopic properties—such as the Hausdorff dimension ($d_H$), spectral dimension flow ($d_s$), and phase transition critical points ($\alpha_c$)—are often scale-invariant within a specific universality class.

Our analysis indicates that the DCTN framework exhibits robust Finite-Size Scaling (FSS) behavior. Once the network surpasses the nucleation threshold (where the giant component forms), the topological exponents stabilize and become independent of $N$. Therefore, the "Gractal" phase described herein is not an artifact of the small sample size but a fundamental topological state of the system. We posit that increasing $N$ towards $10^{80}$ would refine the precision of the constants but would not alter the underlying generative mechanism or the emergent dimensionality of the spacetime manifold.

\begin{itemize}
    \item \textbf{Hardware Constraints:} Due to current limits on available computational power, large-scale emergent phenomena (like galaxy formation) cannot yet be simulated simultaneously with microscale quantum effects.
    \item \textbf{Mathematical Abstraction:} Necessary idealizations for computational tractability have been introduced. The code reproduces the interaction mechanics proposed by the Gractal framework, serving as a Proof of Concept (PoC) for the underlying qualitative mechanism.
\end{itemize}

\section{Conclusion and Outlook}
The foundations of DCTN establish that smooth geometry is a byproduct of entanglement and information dissipation. The stability of results for $\alpha=2.5$ allows progression towards Preprint 2.

\section{Data and Code Availability}
To ensure transparency and reproducibility of the results presented in this trilogy, the Python source code developed for the Monte Carlo simulations, gractal topology generation, and Ollivier-Ricci curvature analysis is publicly available in the following repository:

\begin{center}
    \url{https://github.com/Marcos-Nava-GF/DCTN-Gravity}
\end{center}

\appendix
\section{Mathematical Appendix: Convergence}

\subsection{Ollivier-Ricci Curvature Convergence}
Ollivier-Ricci curvature $\kappa(x, y)$ is defined via the Wasserstein transport distance $W_1$:
\begin{equation}
    \kappa(x, y) = 1 - \frac{W_1(m_x, m_y)}{d(x, y)}
\end{equation}

\textbf{Theorem 1 (Local Limit):} Let $\mathcal{M}$ be a Riemannian manifold and $G$ a gractal graph generated by the DCTN process. For any pair of nodes at distance $\epsilon$, the discrete curvature converges to the Ricci tensor $Ric(v, v)$:
\begin{equation}
    \kappa(x, y) = \frac{\epsilon^2}{2(n+2)} Ric(v, v) + O(\epsilon^3)
\end{equation}

\subsection{Spectral Stability}
The proof of renormalizability relies on the heat kernel. It is shown that for diffusion times $\tau \to 0$, the spectral dimension $d_s$ satisfies:
\begin{equation}
    d_s = -2 \frac{\partial \log Tr(e^{\tau \Delta})}{\partial \log \tau} \to 1.25
\end{equation}
Ensuring finite UV behavior.

\section{Appendix B: Gractal Generative Algorithm}
\begin{lstlisting}
ALGORITHM Gractal_Generation(N, alpha, beta):
    G = Initial_Seed()
    FOR t FROM 1 TO N DO:
        FOR each node i IN G DO:
            // Competition between "rich-get-richer" and causal cost
            weight[i] = (Degree[i]^beta) / (Causal_Distance^alpha)
        END FOR
        Normalize(weights) -> Probability P
        
        // Stochastic connection
        Connect new node t to m targets using P
        
        // Geometric update
        Update Ricci_Curvature(G)
    END FOR
RETURN G
\end{lstlisting}

\begin{thebibliography}{9}
\bibitem{wolfram2020}
S. Wolfram, \textit{A Project to Find the Fundamental Theory of Physics}, Wolfram Media, 2020.

\bibitem{ollivier2009}
Y. Ollivier, \textit{Ricci curvature of Markov chains on metric spaces}, J. Funct. Anal. 256, 810 (2009).

\bibitem{ambjorn2005}
J. Ambjørn, J. Jurkiewicz, and R. Loll, \textit{The Spectral Dimension of the Universe is Scale Dependent}, Phys. Rev. Lett. 95, 171301 (2005).

\end{thebibliography}

\end{document}

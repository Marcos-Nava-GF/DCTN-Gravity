\documentclass[11pt, a4paper]{article}

% Language and Encoding Packages
\usepackage[utf8]{inputenc}
\usepackage[T1]{fontenc}
\usepackage[english]{babel} 

% Math and Physics Packages
\usepackage{amsmath, amssymb, amsthm}
\usepackage{physics}
\usepackage{siunitx}

% Formatting and Graphics Packages
\usepackage{geometry}
\geometry{top=2.5cm, bottom=2.5cm, left=2.5cm, right=2.5cm}
\usepackage{graphicx}
\usepackage{booktabs} 
\usepackage{listings} 
\usepackage{cite}     
\usepackage{hyperref} 
\hypersetup{
    colorlinks=true,
    linkcolor=blue,
    citecolor=blue,
    urlcolor=blue
}

% Code block configuration
\lstset{
basicstyle=\small\ttfamily,
frame=single,
columns=fullflexible,
keepspaces=true,
language=Python 
}

% Document Information
\title{\textbf{Mitigation of the Hubble Tension via UV/IR Dimensional Calibration: A Dynamic Causal Tensor Network Approach}}
\author{\textbf{Marcos Fernando Nava Salazar} \\
\textit{Independent Researcher} \\
Aguascalientes, Mexico}
\date{January 2026 \\ \small{Preprint 2 (v1.5 Final Version)}}

\begin{document}

\maketitle

\begin{abstract}
The persistent discrepancy between the Hubble constant ($H_0$) derived from the early Universe ($\Lambda$CDM) and local observations (SH0ES) suggests a breakdown in the standard cosmological metric. We propose a solution based on the Gractal framework, postulating that the expansion rate is thermodynamically modulated by the local information density of the cosmic web. We derive a density-dependent efficiency scaling law, where the filamentous topology ($d_H \approx 1.41$) enhances information dissipation in under-dense regions (voids). This mechanism yields a mean dimensional boost of $\alpha_a \approx +2.5\%$ in the local volume, effectively reconciling the Planck value ($67.88$ km s$^{-1}$ Mpc$^{-1}$) with local measurements ($70.33$ km s$^{-1}$ Mpc$^{-1}$, reducing tension to $\approx 2.5\sigma$) while strictly preserving the chronological consistency of the universe ($t_0 \approx 13.72$ Ga) relative to stellar constraints (HD 140283).
\end{abstract}

\section{Introduction}
Standard cosmology assumes a rigid expansion metric. However, if spacetime emerges from a Dynamic Causal Tensor Network (DCTN), expansion is a thermodynamic process. We formalize this via the relation:
\begin{equation}
    E_a = E_i + \delta_H
\label{eq:expansion_formula}
\end{equation}
Where:
\begin{itemize}
    \item $E_i$ (\textbf{Initial Expansion}): The base value from the early Universe ($\Lambda$CDM/Planck), $E_i \approx 67.88 \text{ km s}^{-1}\text{Mpc}^{-1}$.
    \item $\delta_H$ (\textbf{Hubble Correction / Dimensional Perturbation}): The distinctive boost from the gractal network's efficiency in dissipating information at the IR scale.
    \item $E_a$ (\textbf{Actual Expansion}): The locally observed rate, converging towards SH0ES data.
\end{itemize}

\section{Thermodynamics of Gractal Expansion}

\subsection{Information Dissipation Efficiency}
Expansion is driven by the creation of new network links, each dissipating information $Q \ge k_{B} T \ln 2$. The parameter $\delta_H$ quantifies the increased efficiency of this process in a filamentous geometry compared to a flat Euclidean vacuum.

\subsection{The Optimal Thermodynamic Efficiency Point}
Our simulations identify $\delta_H = +2.5\%$ as the critical calibration value corresponding to maximum structural stability in the network.
\begin{itemize}
    \item \textbf{Base ($E_i$)}: $67.88 \text{ km s}^{-1}\text{Mpc}^{-1}$
    \item \textbf{Boost ($\delta_H$)}: $+2.5\%$ (Empirically derived stability peak)
    \item \textbf{Result ($E_a$)}: $70.33 \text{ km s}^{-1}\text{Mpc}^{-1}$
\end{itemize}
This correction reduces the tension with SH0ES ($75.26 \text{ km s}^{-1}\text{Mpc}^{-1}$) from a critical $7.57$ difference to a manageable $4.93 \text{ km s}^{-1}\text{Mpc}^{-1}$ ($\approx 2.5\sigma$).

\section{Dynamic Anisotropy: Phenomenological Density Scaling}
Beyond a static constant, we postulate that $\delta_H$ functions as a dynamic scalar field dependent on the local information density $\rho$, derived from network saturation efficiency. This approach models the universe not as a static manifold but as a computational fluid where expansion rates are locally regulated by complexity.

\subsection{Density-Dependent Efficiency Ansatz}
In a filamentous network topology characterized by a Hausdorff dimension $d_H \approx 1.41$, the efficiency of spatial emergence is inversely proportional to the nodal connectivity density. We propose a phenomenological scaling law where the effective expansion rate $E_a(\rho)$ is modulated by the local density profile:
\begin{equation}
E_{a}(\rho)=E_{i}\left(1+\frac{\delta_{base}}{\rho^{2-d_{H}}}\right)
\end{equation}
This relation implies a divergence in expansion rates based on local structure:
\begin{itemize}
    \item \textbf{Under-dense Regions (Voids, $\rho < 1$):} In cosmic voids, the low node density minimizes information friction, maximizing the dimensional calibration factor ($\delta_H \uparrow$). This results in a locally enhanced expansion rate ($H_0 \rightarrow 75+ \text{ km s}^{-1} \text{Mpc}^{-1}$), consistent with SH0ES measurements which are sensitive to the local, void-dominated geometry.
    \item \textbf{Over-dense Regions (Hubs, $\rho > 1$):} In virialized structures like galaxy clusters, high interconnectivity leads to information saturation ($\delta_H \downarrow$). The expansion rate asymptotically reverts to the inertial Planck base value ($E_i \approx 67.88 \text{ km s}^{-1} \text{Mpc}^{-1}$), recovering the $\Lambda$CDM prediction in high-density environments.
\end{itemize}

\begin{figure}[h!]
    \centering
    \includegraphics[width=0.8\textwidth]{images/p2_fig2_expansion_vs_density.png}
    \caption{Variation of Expansion Rate $E_a$ vs Local Density. The model predicts higher expansion in cosmic voids and convergence to $\Lambda$CDM in dense clusters.}
    \label{fig:napkin_logic}
\end{figure}

\section{Physical Constraints and Dimensional Topology}

\subsection{Cosmic Age Limit (Chronological Consistency)}
Any modification to $H_0$ impacts the age of the universe $t_0$. We strictly validate our model against the oldest known star, HD 140283 ("Methuselah"), dated at $\approx 14.46 \pm 0.8$ Ga (lower bound limit $\sim 13.6$ Ga).
\begin{itemize}
    \item A $+4.5\%$ adjustment would yield $t_0 < 13.6$ Ga, violating stellar physics.
    \item Our selected $+2.5\%$ yields **$t_0 \approx 13.72$ Ga**, maintaining physical validity.
\end{itemize}

\subsection{Topological Justification ($d_H \approx 1.41$)}
The thermodynamic boost $\delta_H$ is rooted in the network's topology.
\begin{enumerate}
    \item \textbf{Hausdorff Dimension ($d_H \approx 1.41$)}: At fundamental scales, spacetime is not a solid block but a filamentous web. This sub-dimensional scaling ($R^{1.41}$ vs $R^3$) implies that "filling" space requires less energy, or conversely, the same energy creates space more efficiently ($+2.5\%$ efficiency).
    \item \textbf{Spectral Flow ($d_s$)}: The transition from $d_s \approx 1.25$ (UV) to $d_s \approx 4.0$ (IR) acts as a regulator. At the Plank scale, the network is highly connected and sub-diffusive, preventing UV divergences.
    \item \textbf{Minkowski Convergence}: As the network grows, local curvature averages out ($K \approx 0.03$), recovering smooth Einsteinian gravity in the limit.
\end{enumerate}

\section{Simulation Results}
\begin{figure}[h!]
    \centering
    \includegraphics[width=0.8\textwidth]{images/p2_fig1_hubble_calibration.png}
    \caption{Calibration of $\delta_H$. The blue line shows the increase in $E_a$, while the red line tracks the decreasing Age of the Universe. The purple line marks the 2.5\% Sweet Spot where tension is minimized without violating the Methuselah limit.}
    \label{fig:calibration}
\end{figure}

Data provided in Table \ref{tab:calibration} confirms the robustness of the $\delta_H = 2.5\%$ solution.

\begin{table}[h]
    \centering
    \caption{Correction Summary}
    \label{tab:calibration}
    \begin{tabular}{lccc}
        \toprule
        Parameter & Base (Planck) & Adjusted ($+2.5\%$) & Target (SH0ES) \\
        \midrule
        $H_0$ (km/s/Mpc) & 67.88 & \textbf{70.33} & 75.26 \\
        Tension ($\sigma$) & $>5\sigma$ & $\mathbf{\approx 2.5\sigma}$ & - \\
        \bottomrule
    \end{tabular}
\end{table}

\section{Data Availability}
Code available at \url{https://github.com/Marcos-Nava-GF/DCTN-Gravity}. Simulation scripts located in \texttt{Simulations/Hubble\_Tension}.

\end{document}
